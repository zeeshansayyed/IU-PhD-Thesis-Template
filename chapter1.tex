\chapter{Introduction and Background}

This is only a placeholder. This chapter should be written at the very end.

Parsing is an important problem in natural language processing. It involves deducing the linguistic structure of a given piece of text or sentence. In fact, parsing is a very broad term and can refer to its many different types, namely: syntactic, semantic or discourse. While semantic parsing deals with the representation of the meaning of the given piece of text, discourse parsing attempts to capture the relations between the different sentences of the text. Syntactic parsing, on the other hand, involves analyzing and assigning syntactic structure to a sentence in the form of syntactic representations called parse trees. The focus of thesis will chiefly be on syntactic parsing and its associated problems.

Syntactic parsing is a fundamental task of natural language processing, which forms a base on which many secondary tasks are built. Parse trees or syntactic representations produced by syntactic parsing can be used directly in tasks such as grammar checking. Moreover, they are used as intermediate structures in a variety of other tasks like machine translation, semantic analysis, information retrieval and question answering. For instance, to answer a wh-question like `\textit{Who did this?}' based on a given sentence, it is important to know what is the subject or the doer of the activity in that sentence. 

A corpus in which every sentence is syntactically annotated with a parse tree is called a treebank. A treebank typically consists of thousands of sentences drawn from a variety of sources, which are manually annotated with correct parse trees by humans. A wide variety of treebanks have been created for various languages like English, Arabic, Chinese, Czech, etc. The Penn Treebank Project, for instance, has treebanks from the Brown, Switchboard, Air Traffic Information System(ATIS) and the Wall Street Journal corpora of English \cite{jurafsky2014speech}. These treebanks play an important role in empirical investigations of various statistical phenomena and have considerably helped in the development of many parsers and parsing techniques over the past two decades. In fact, manually annotated treebanks allow syntactic parsing to be treated as a supervised machine learning problem. The sentences act as the independent variable $(X \in \mathcal{X})$ which are mapped to the parse trees, which act as the target or dependent variable $(y \in Y)$. Since the parse trees represent a complex structure and not just a simple class label or a number, parsing is actually viewed as structured prediction problem. The goal of this problem find the mapping function $f: \mathcal{X} \rightarrow \mathcal{Y}$.

